%%%%%%%%%%%%%%%%%%%%%%%%%%%%%%%%%%%%%%%%%%%%%%%%%%%%%%%%%%%%%%%%%%%%%%%%%%%%%%%%%%%%%%%%%%%%%%%%
%
% CS484 Written Question Template
%
% Acknowledgements:
% The original code is written by Prof. James Tompkin (james_tompkin@brown.edu).
% The second version is revised by Prof. Min H. Kim (minhkim@kaist.ac.kr).
%
% This is a LaTeX document. LaTeX is a markup language for producing 
% documents. Your task is to fill out this document, then to compile 
% it into a PDF document. 
%
% 
% TO COMPILE:
% > pdflatex thisfile.tex
%
% If you do not have LaTeX and need a LaTeX distribution:
% - Personal laptops (all common OS): www.latex-project.org/get/
% - We recommend latex compiler miktex (https://miktex.org/) for windows,
%   macTex (http://www.tug.org/mactex/) for macOS users.
%   And TeXstudio(http://www.texstudio.org/) for latex editor.
%   You should install both compiler and editor for editing latex.
%   The another option is Overleaf (https://www.overleaf.com/) which is 
%   an online latex editor.
%
% If you need help with LaTeX, please come to office hours. 
% Or, there is plenty of help online:
% https://en.wikibooks.org/wiki/LaTeX
%
% Good luck!
% Min and the CS484 staff
%
%%%%%%%%%%%%%%%%%%%%%%%%%%%%%%%%%%%%%%%%%%%%%%%%%%%%%%%%%%%%%%%%%%%%%%%%%%%%%%%%%%%%%%%%%%%%%%%%
%
% How to include two graphics on the same line:
% 
% \includegraphics[width=0.49\linewidth]{yourgraphic1.png}
% \includegraphics[width=0.49\linewidth]{yourgraphic2.png}
%
% How to include equations:
%
% \begin{equation}
% y = mx+c
% \end{equation}
% 
%%%%%%%%%%%%%%%%%%%%%%%%%%%%%%%%%%%%%%%%%%%%%%%%%%%%%%%%%%%%%%%%%%%%%%%%%%%%%%%%%%%%%%%%%%%%%%%%

\documentclass[11pt]{article}

\usepackage[english]{babel}
\usepackage[utf8]{inputenc}
\usepackage[colorlinks = true,
linkcolor = blue,
urlcolor  = blue]{hyperref}
\usepackage[a4paper,margin=1.5in]{geometry}
\usepackage{stackengine,graphicx}
\usepackage{fancyhdr}
\setlength{\headheight}{15pt}
\usepackage{microtype}
\usepackage{times}

% From https://ctan.org/pkg/matlab-prettifier
\usepackage[numbered,framed]{matlab-prettifier}

\frenchspacing
\setlength{\parindent}{0cm} % Default is 15pt.
\setlength{\parskip}{0.3cm plus1mm minus1mm}

\pagestyle{fancy}
\fancyhf{}
\lhead{Homework 2 Questions}
\rhead{CS484}
\rfoot{\thepage}

\date{}

\title{\vspace{-1cm}Homework 2 Questions}


\begin{document}
	\maketitle
	\vspace{-3cm}
	\thispagestyle{fancy}
	
	\section*{Instructions}
	\begin{itemize}
		\item 4 questions.
		\item Write code where appropriate.
		\item Feel free to include images or equations.
		\item Please make this document anonymous.
		\item \textbf{Please use only the space provided and keep the page breaks.} Please do not make new pages, nor remove pages. The document is a template to help grading.
		\item If you really need extra space, please use new pages at the end of the document and refer us to it in your answers.
	\end{itemize}

	\section*{Questions}
	
	\paragraph{Q1:} Explicitly describe image convolution: the input, the transformation, and the output. Why is it useful for computer vision?
	
	%%%%%%%%%%%%%%%%%%%%%%%%%%%%%%%%%%%
	\paragraph{A1:} Your answer here.
	
	\begin{itemize}
    	\item input: Original image, kernel.
    	\item transformation: Flip both the rows and columns of the kernel. Then, for each entries in the image, multiply each kernel entries by the corresponding image matrix entries(local neighbors of the current entry), and sum it to get the value of the entry of the new image.
    	\item output: New modified image.
    \end{itemize}
    
    Convolution is useful for computer vision because: \\
    Using convolution, we can make operators whose output pixel values are linear combinations of input pixel values (the corresponding input pixel itself and its neighbors). This enables image processing by filtering - such as feature extraction.
	
	
	%%%%%%%%%%%%%%%%%%%%%%%%%%%%%%%%%%%
	
	% Please leave the pagebreak
	\pagebreak
	\paragraph{Q2:} What is the difference between convolution and correlation? Construct a scenario which produces a different output between both operations.
	
	\emph{Please use \href{https://www.mathworks.com/help/images/ref/imfilter.html}{$imfilter$} to experiment! Look at the `options' parameter in MATLAB Help to learn how to switch the underlying operation from correlation to convolution.}
	
	%%%%%%%%%%%%%%%%%%%%%%%%%%%%%%%%%%%
	\paragraph{A2:} Your answer here.
	
	Theoretical meaning difference: Convolution is performing linear operations on the signal (for our case - 2D signal - image), and correlation is a measure of similarity between two signals. \\
	For both operations, kernel matrix slides through the original image matrix and apply multiplication to each entry and its neighbors. However, in convolution, the matrix is rotated by 180 degrees. \\
	Case example: \\
	\begin{lstlisting}[style=Matlab-editor]
A = [1, 2, 3, 4; 5, 6, 7, 8; 9, 10, 11, 12]
F = [-10, 0, 10; -1, 0, 1; -0.1, 0, 0.1]

imfilter(A, F) % correlation
imfilter(A, F, 'conv') % convolution
    \end{lstlisting}
    Output is
    \begin{lstlisting}
A =
     1     2     3     4
     5     6     7     8
     9    10    11    12
F =
  -10.0000         0   10.0000
   -1.0000         0    1.0000
   -0.1000         0    0.1000
ans =
    2.6000    2.2000    2.2000   -3.7000
   27.0000   22.2000   22.2000  -38.1000
   70.0000   22.0000   22.0000  -81.0000
ans =
  -62.0000  -22.0000  -22.0000   73.0000
 -106.2000  -22.2000  -22.2000  117.3000
  -10.6000   -2.2000   -2.2000   11.7000
    \end{lstlisting}
    
    Element (1, 1) for example, \\
    First(correlation) ans(1, 1) is equal to A(0, 0) * F(0, 0) + ... + A(2, 2) * F(2, 2), \\
    while second(convolution) ans(1, 1) is equal to A(0, 0) * F(2, 2) + ... + A(2, 2) * F(0, 0).
	%%%%%%%%%%%%%%%%%%%%%%%%%%%%%%%%%%%
	
	% Please leave the pagebreak
	\pagebreak
	\paragraph{Q3:} What is the difference between a high pass filter and a low pass filter in how they are constructed, and what they do to the image? Please provide example kernels and output images.
	
	%%%%%%%%%%%%%%%%%%%%%%%%%%%%%%%%%%%
	\paragraph{A3:} Your answer here.
	\\
	Low pass filter smooths an image by decreasing the disparity between pixel values. The filter (kernel) is constructed in a way that it retains the low frequency information within an image while reducing the high frequency information, averaging nearby pixels. \\
	
	High pass filter sharpens an image by enhancing the contrast between adjoining areas with littel variation in intensity. The filter is constructed in a way that it retains the high frequency information within an image while reducing the low frequency information, by increasing the intesity of the center pixel relative to neighboring pixels. \\
	
	Below is the example kernels and output filtered images of low pass filter and high pass filter.
	\\
	
	Original image: 
	\href{plane.bmp}{plane.bmp}
	\\
	
    Low pass filter example:
    \begin{lstlisting}
    1/9            1/9            1/9     
    1/9            1/9            1/9     
    1/9            1/9            1/9
    \end{lstlisting}
    Low pass filtered image:
	\href{lpf_img.jpg}{lpf\_img.jpg}
	\\
	
	High pass filter example:
    \begin{lstlisting}
    -1/9           -1/9           -1/9     
    -1/9            8/9           -1/9     
    -1/9           -1/9           -1/9
    \end{lstlisting}
    High pass filtered image:
	\href{hpf_img.jpg}{hpf\_img.jpg} \\
	Note that the attached high pass filtered image file is an image made by adding +0.5 to each pixel values from the real high pass filtered result. (This is because the real result is centered around zero and visualizing it will result in mostly black.)
	%%%%%%%%%%%%%%%%%%%%%%%%%%%%%%%%%%%
	
	% Please leave the pagebreak
	\pagebreak
	\paragraph{Q4:} How does computation time vary with filter sizes from $3\times3$ to $15\times15$ (for all odd and square sizes), and with image sizes from 0.25~MPix to 8~MPix (choose your own intervals)? Measure both using \href{https://www.mathworks.com/help/images/ref/imfilter.html}{$imfilter$} to produce a matrix of values. Use the \href{https://www.mathworks.com/help/images/ref/imresize.html}{$imresize$} function to vary the size of an image. Use an appropriate charting function to plot your matrix of results, such as \href{https://www.mathworks.com/help/matlab/ref/scatter3.html}{$scatter3$} or \href{https://www.mathworks.com/help/matlab/ref/surf.html}{$surf$}.
	
	Do the results match your expectation given the number of multiply and add operations in convolution?
	
	See RISDance.jpg in the attached file.
	
	%%%%%%%%%%%%%%%%%%%%%%%%%%%%%%%%%%%
	\paragraph{A4:} Your answer here.
	
	The Comparison was done for filter sizes $3\times3$, $5\times5$, ..., $15\times15$ and image resized to 10 steps between 0.25MPix and 8MPix, by code below.
	\begin{lstlisting}[style=Matlab-editor]
img = im2single(imread('../questions/RISDance.jpg'));
img_pixels = linspace(0.25 * 1e+6, 8 * 1e+6, 10);
filt_sizes = 3 : 2 : 15;
filt_pixels = zeros(numel(filt_sizes), 1);
filts = cell(numel(filt_sizes));
for fidx = 1 : numel(filt_sizes)
    filts{fidx} = fspecial('average', filt_sizes(fidx));
    filt_pixels(fidx) = numel(filts{fidx});
end

results = zeros(numel(img_pixels), numel(filt_sizes));
img_actual_pixels = zeros(numel(img_pixels), 1);
for pidx = 1 : numel(img_pixels)
    pnum = img_pixels(pidx);
    height = round(sqrt(pnum / 2));
    img_actual_pixels(pidx) = 2 * height * height;
    resized_img = imresize(img, [height, 2 * height]);
    for fidx = 1 : numel(filt_sizes)
        t = clock;
        imfilter(resized_img, filts{fidx});
        results(pidx, fidx) = etime(clock, t) * 1000;
    end
end
% Plot X:filt_pixels, Y:img_actual_pixels, Z:results
	\end{lstlisting}
	Plotted result : \href{surf.jpg}{surf.jpg} \\
	If matlab imfilter function is using the simple iterative method for convolution, number of multiply and add operations is proportional to $img\_actual\_pixels\times filt\_pixels$. The time result mostly followed this, except that for $3\times3$ filter and $5\times5$ filter, the time decreased, which was different from the expectation.
	
	%%%%%%%%%%%%%%%%%%%%%%%%%%%%%%%%%%%
	
	
	% If you really need extra space, uncomment here and use extra pages after the last question.
	% Please refer here in your original answer. Thanks!
	%\pagebreak
	%\paragraph{AX.X Continued:} Your answer continued here.
	
	
	
\end{document}