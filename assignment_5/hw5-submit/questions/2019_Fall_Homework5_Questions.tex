%%%%%%%%%%%%%%%%%%%%%%%%%%%%%%%%%%%%%%%%%%%%%%%%%%%%%%%%%%%%%%%%%%%%%%%%%%%%%%%%%%%%%%%%%%%%%%%%
%
% CS484 Written Question Template
%
% Acknowledgements:
% The original code is written by Prof. James Tompkin (james_tompkin@brown.edu).
% The second version is revised by Prof. Min H. Kim (minhkim@kaist.ac.kr).
%
% This is a LaTeX document. LaTeX is a markup language for producing 
% documents. Your task is to fill out this document, then to compile 
% it into a PDF document. 
%
% 
% TO COMPILE:
% > pdflatex thisfile.tex
%
% If you do not have LaTeX and need a LaTeX distribution:
% - Personal laptops (all common OS): www.latex-project.org/get/
% - We recommend latex compiler miktex (https://miktex.org/) for windows,
%   macTex (http://www.tug.org/mactex/) for macOS users.
%   And TeXstudio(http://www.texstudio.org/) for latex editor.
%   You should install both compiler and editor for editing latex.
%   The another option is Overleaf (https://www.overleaf.com/) which is 
%   an online latex editor.
%
% If you need help with LaTeX, please come to office hours. 
% Or, there is plenty of help online:
% https://en.wikibooks.org/wiki/LaTeX
%
% Good luck!
% Min and the CS484 staff
%
%%%%%%%%%%%%%%%%%%%%%%%%%%%%%%%%%%%%%%%%%%%%%%%%%%%%%%%%%%%%%%%%%%%%%%%%%%%%%%%%%%%%%%%%%%%%%%%%
%
% How to include two graphics on the same line:
% 
% \includegraphics[width=0.49\linewidth]{yourgraphic1.png}
% \includegraphics[width=0.49\linewidth]{yourgraphic2.png}
%
% How to include equations:
%
% \begin{equation}
% y = mx+c
% \end{equation}
% 
%%%%%%%%%%%%%%%%%%%%%%%%%%%%%%%%%%%%%%%%%%%%%%%%%%%%%%%%%%%%%%%%%%%%%%%%%%%%%%%%%%%%%%%%%%%%%%%%

\documentclass[11pt]{article}

\usepackage[english]{babel}
\usepackage[utf8]{inputenc}
\usepackage[colorlinks = true,
            linkcolor = blue,
            urlcolor  = blue]{hyperref}
\usepackage[a4paper,margin=1.5in]{geometry}
\usepackage{stackengine,graphicx}
\usepackage{fancyhdr}
\setlength{\headheight}{15pt}
\usepackage{microtype}
\usepackage{times}

% From https://ctan.org/pkg/matlab-prettifier
\usepackage[numbered,framed]{matlab-prettifier}

\frenchspacing
\setlength{\parindent}{0cm} % Default is 15pt.
\setlength{\parskip}{0.3cm plus1mm minus1mm}

\pagestyle{fancy}
\fancyhf{}
\lhead{Homework 5 Questions}
\rhead{CS484}
\rfoot{\thepage}

\date{}

\title{\vspace{-1.5cm}Homework 5 Questions}


\begin{document}
\maketitle
\vspace{-3cm}
\thispagestyle{fancy}

\section*{Instructions}
\begin{itemize}
  \item 3 questions.
  \item Write code where appropriate.
  \item Feel free to include images or equations.
  \item Please make this document anonymous.
  \item \textbf{Please use only the space provided and keep the page breaks.} Please do not make new pages, nor remove pages. The document is a template to help grading. If you need extra space, please use and refer to new pages at the end of the document.
\end{itemize}

\section*{Questions}

\paragraph{Q1:} Given a linear classifier, how might we handle data that are not linearly separable? How does the \emph{kernel trick} help in these cases? (See course slides in supervised learning, plus your own research.)

%%%%%%%%%%%%%%%%%%%%%%%%%%%%%%%%%%%
\paragraph{A1:} Your answer here.

For data that are not linearly separable, we can map the data to a space of higher dimension. By mapping the original input space to some higher dimensional feature space, the training data can be linearly separable. \\
This strategy is called kernel trick. When using this strategy, instead of explicitly computing the lifting transformation $\phi(x)$, we can define a kernel function $K$ such that
$$
K(x_i, x_j) = \phi(x_i) \cdot \phi(x_j)
$$
By this, we can have a transformed feature space. The classifier is a hyperplane in the transformed feature space and it can be nonlinear in the original input space. \\
Some common kernels include homogeneous polynomial, inhomogeneous polynomial, Gaussian radial basis function, and hyperbolic tangent.

%%%%%%%%%%%%%%%%%%%%%%%%%%%%%%%%%%%

\pagebreak
\paragraph{Q2:} In machine learning, what are bias and variance? When we evaluate a classifier, what are overfitting and underfitting, and how do these relate to bias and variance?

%%%%%%%%%%%%%%%%%%%%%%%%%%%%%%%%%%%
\paragraph{A2:} Your answer here.

In machine learning,

\textbf{Bias} is a difference between the expected (or average) prediction of our model and the correct value. Bias error is an error from erroneous assumptions and simplifications in the learning algorithm. \\
\textbf{Variance} is an error from sensitivity to small fluctuations in the training set. It is an amount that the estimate of the target funciton will change if different training data is used.

\textbf{Overfitting} means that the model corresponds too closely or exactly to a particular set of data, so that it may fail to fit additional data or predict future observations reliably. This means that the model is too complex and fits irrelevant characteristics like noise in the data. \textbf{High variance} can make model sensitive to random noise in training data, and therefore cause overfitting. \\
\textbf{Underfitting} means that the model cannot capture the underlying structure of the data properly. This means that model is too simple to represent all the relevant class characteristics. In other words, some of the parameters that should appear in an adequate model are missing. \textbf{High bias} can make model miss the relevant relations between input features and output target, and therefore cause underfitting.

%%%%%%%%%%%%%%%%%%%%%%%%%%%%%%%%%%%

% Please leave the pagebreak
\pagebreak
\paragraph{Q3:} The way that the bag of words representation handles the spatial layout of visual information can be both an advantage and a disadvantage. Describe an example scenario for each of these cases, plus describe a modification or additional algorithm which can overcome the disadvantage. 

How might we evaluate whether bag of words is a good model?

%%%%%%%%%%%%%%%%%%%%%%%%%%%%%%%%%%%
\paragraph{A3:} Your answer here.

\textbf{Advantage}. Main advantage of bag of words representation is that it can extract local features of images with invariance to scale and orientation, if we use methods such as Harris corner detector, SIFT detector, SURF detector, etc, for local features. Also, bag of words representation can be robust to partial occlusion, unlike the global appearance approaches. So, bag of words representation has good performance for recognition of images with lots of variations - images of animals, for example. \\
\textbf{Disadvantage}. Since bag of words representation describes image by histograms of features, it ignores the spatial relationships between the local feature patches. So, bag of words representation's performance can be very poor for recognition of images of fixed shape - images of certain type of vehicle, for example. \\
\textbf{Modification}. In order to overcome the disadvantage described above, several methods that can incorporate the spatial information of features have been proposed. One example is spatial pyramid matching. In this method, the image is repeatedly subdivided, creating multi-level image partitions. In each subdividing steps, histograms of image features are computed over the resulting sub-regions. Image recognition is performed by comparing the collection of these histograms, and in this way the spatial information of features are not ignored.

%%%%%%%%%%%%%%%%%%%%%%%%%%%%%%%%%%%

% If you really need extra space, uncomment here and use extra pages after the last question.
% Please refer here in your original answer. Thanks!
%\pagebreak
%\paragraph{AX.X Continued:} Your answer continued here.



\end{document}
